% ========== RESULTADOS ==========
% Resultados (valor de 1.0 puntos)
\section{Resultados}
La experiencia de laboratorio fue separada en cuatro partes principales, de las cuales, los resultados serian
reagrupados para su eventual uso.\\

Inicialmente, se busco la familiarizacion con la instrumentacion que fue utilizada en la experiencia, para lo cual
fue montado el equipamiento cuya principal enfoque era examinar el comportamiento del periodo ($T$) y la
frecuencia ($f$) dada una señal triangular generada. Los resultados obtenidos de esta examinacion fueron
respectivamentes reagrupados en la tabla 1 (ver: Tabla 1), en las cuales se exponen el periodo y la frecuencia
nombrados junto a su error asociado.\\
{ %Tabla 1
\begin{table}
    \begin{tabular}{|c|c|c|c|}
     \hline
         Periodo (T) $[ms]$ & Error periodo $\Delta T$ & Frecuencia (f) $[Hz]$ & Error frecuencia $\Delta f$\\ \hline
         2.000 & 0.0005 & 500.0 & chamullo \\ \hline
     \end{tabular}\\
     \caption{Periodo, frecuencia y su error asociado}
\end{table}
}

Siguiendo el estudio del equipamiento usado, para indagar el "\it rango de validez \rm" del multímetro, se realizo
la conexion de este ultimo a una emisora de frecuencias y osciloscopio utilizando los cables $BNC$-banana. De esta
manera, emitiendo dos señales de voltajes alternos, una sinusoidal y la otra triangular. De esta forma se vario las
frecuencias emitidas en el rango propuesto de $[50[Hz], 2000[Hz]]$ con el fin de registrar el comportamiento del
voltaje $[V]$ registrado por la instrumentacion. Estos resultados pueden verse en las tablas 2 y 3 (ver: Tabla 2 y
Tabla 3) para las 10 mediciones realizadas.
{% Mediciones: Tablas y graficos
\begin{table}
    \begin{tabular}{|c|c|c|c|}
     \hline
        n° Medicion & Frecuencia emitida $[Hz]$ & Voltaje medido $[V]$ & Error Voltaje $\Delta V$\\ \hline
        1 & 50 & 0.718 & 0.0005 \\
        2 & 250 & 0.716 & 0.0005 \\
        3 & 450 & 0.709 & 0.0005 \\
        4 & 650 & 0.699 & 0.0005 \\
        5 & 850 & 0.686 & 0.0005 \\
        6 & 1050 & 0.671 & 0.0005 \\
        7 & 1250 & 0.655 & 0.0005 \\
        8 & 1450 & 0.636 & 0.0005 \\
        9 & 1650 & 0.617 & 0.0005 \\
        10 & 1850 & 0.598 & 0.0005 \\ \hline
    \end{tabular}
    \caption{Mediciones emisión sinusoidal}
\end{table}

\begin{table}
    \begin{tabular}{|c|c|c|c|}
     \hline
     n° Medicion & Frecuencia emitida $[Hz]$ & Voltaje medido $[V]$ & Error Voltaje $\Delta V$\\ \hline
     1 & 50 & 0.559 & 0.0005 \\
     2 & 250 & 0.557 & 0.0005 \\
     3 & 450 & 0.553 & 0.0005 \\
     4 & 650 & 0.547 & 0.0005 \\
     5 & 850 & 0.538 & 0.0005 \\
     6 & 1050 & 0.528 & 0.0005 \\
     7 & 1250 & 0.515 & 0.0005 \\
     8 & 1450 & 0.503 & 0.0005 \\
     9 & 1650 & 0.489 & 0.0005 \\
     10 & 1850 & 0.474 & 0.0005 \\
     extra & 2000 & 0.464 & 0.0005 \\ \hline
    \end{tabular}
    \caption{Mediciones emisión triangular}
\end{table}
}
El estudio de las mediciones de voltaje $[V]$ conto con un bosquejo agregado con el fin de referenciar el
comportamiento previamente dados (ver: Figura X). Ademas de los resultados incorporados en la tabla 4
(ver: Tabla 4), que contiene los resultados equivalentes a las mediciones de mayor y menor frecuencia posibles
(es decir, $2000[Hz]$ y $50[Hz]$), los cuales no fueron agregados previamente pero, se considero oportuno su mencion.
{
\begin{figure}[h!]
    \centering
        \subfloat{}{
         \label{f:Sinusoidal}
          \includegraphics[width = 5cm]{imagenes/grafico sinusoidal.png}}
        \subfloat{}{
         \label{f:Triangular}
          \includegraphics[width = 5cm]{imagenes/grafico triangular.png}}
        \subfloat{}{
         \label{f:ambas}
          \includegraphics[width = 5cm]{imagenes/ambos dos.png}}
    \caption{Gráficos para mediciones}
    \label{fig:enter-label}
\end{figure}

\begin{table}[h!]
    \begin{tabular}{|l|c|c|c|}
     \hline
     Casos & Frecuencia emitida $[Hz]$ & Voltaje medido $[V]$ & Error Voltaje $\Delta V$  \\ \hline
     Caso inicial (sinusoidal) & 50 & 0.718 & 0.0005 \\
     Caso final  & 2000 & 0.583 & 0.0005 \\ \hline
     Caso inicial (triangular) & 50 & 0.559 & 0.0005 \\
     Caso final  & 2000 & 0.464 & 0.0005 \\\hline
    \end{tabular}
    \caption{Mediciones emisión sinusoidal y triangular, casos extremos}
\end{table}
}

Finalizando la experiencia, se montaron dos circuitos, de los cuales el primero seria uno con conexion en RC
(resistencia-condensador) y, el segundo compuesto por una inductancia reemplazando al condensador, es decir, en RL.
En este caso, se tiene que la tabla 5 (ver: Tabla 5) representa las constantes de tiempo $\tau$ de un condensador o,
un circuito en RL respctivamente para esta situacion.
\begin{table}
    \begin{tabular}{|c|c|c|}
        \hline
        Circuito & Calculo & Resultado (s^{-1})\\ \hline
        RC & $RC$ & $\aprox 10^{-5}$\\
        RL & $\frac{L}{R}$ & $2.23\cdot 10^{-4}$ \\
    \end{tabular} \\
\end{table}


Finalmente, al circuito de estudio se le agregaría una inductancia de $22[mH]$ retirando el condensador
inicialmente conectado. Sin embargo, esta contaría con una resistencia propia que difiere de los parámetros
otorgados inicialmente por el condensador. Debido a esto, resulto oportuno medir el valor, el cual quedaría
dado por el valor siguiente (ver: Tabla 5): \\
{% Tabla de inductancia
\begin{table}
    \begin{tabular}{|c|c|}
     \hline
        Inductancia $[mH]$ & Resistencia $[\Omega]$ \\ \hline
        223.0 & 89.8  \\ \hline
     \end{tabular}
     \caption{Resistencia dada la inductancia}
\end{table}
}



{\LARGE (RESULTADOS PARTE 4)} \\
