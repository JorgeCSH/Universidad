% Template:     Informe/Reporte LaTeX
% Documento:    Archivo principal
% Versión:      5.5.3 (05/09/2018)
% Codificación: UTF-8
%
% Autor: Pablo Pizarro R. @ppizarror
%        Facultad de Ciencias Físicas y Matemáticas
%        Universidad de Chile
%        pablo.pizarro@ing.uchile.cl, ppizarror.com
%
% Manual template: [http://latex.ppizarror.com/Template-Informe/]
% Licencia MIT:    [https://opensource.org/licenses/MIT/]

\documentclass[letterpaper,11pt]{article}
\usepackage[utf8]{inputenc}
\usepackage{xcolor}
\usepackage{listings}

\definecolor{codeblue}{HTML}{1F1F2E} % Color azul marino
\definecolor{codegray}{rgb}{0.5,0.5,0.5}
\definecolor{codepurple}{rgb}{0.58,0,0.82}
\definecolor{backcolour}{rgb}{0.95,0.95,0.92}

\lstdefinestyle{mystyle}{
    backgroundcolor=\color{codeblue},
    commentstyle=\color{codegray},
    keywordstyle=\color{magenta},
    numberstyle=\tiny\color{codegray},
    stringstyle=\color{codepurple},
    basicstyle=\ttfamily\color{white},
    breakatwhitespace=false,
    breaklines=true,
    captionpos=b,
    keepspaces=true,
    numbers=left,
    numbersep=5pt,
    showspaces=false,
    showstringspaces=false,
    showtabs=false,
    tabsize=2
}




% INFORMACIÓN DEL DOCUMENTO Y DATOS PARA PORTADA
% Para cambiar el estilo de la portada y los encabezados-pie de página entrar a la carpeta lib y el archivo config.tex (lib\config.tex) y modificar las lineas 33 y 32 (funciones \portraitstyle y \hfstyle) cambiando el número del parámetro styleX.
% Importante al cambiar el estilo de portada VERIFICAR que esta cumpla los requisitos formales.
\def\titulodelinforme {Tarea Numérica N°2}
\def\temaatratar {Redes Neuronales Artificiales 2}

\def\autordeldocumento {Francisco González}
\def\nombredelcurso {Cálculo en Varias Variables}
\def\codigodelcurso {MA2001-5}

\def\nombreuniversidad {Universidad de Chile}
\def\nombrefacultad {Facultad de Ciencias Físicas y Matemáticas}
\def\departamentouniversidad {Departamento de Ingeniería Matemática}
\def\imagendepartamento {departamentos/fcfm}
\def\imagendepartamentoescala {0.2}
\def\localizacionuniversidad {Santiago, Chile}

\def\tablaintegrantes {
\begin{tabular}{ll}
	Autor:
	& \begin{tabular}[t]{@{}l@{}}
		Francisco González U. \\ 
	\end{tabular} \\
	Profesor:
	& \begin{tabular}[t]{@{}l@{}}
		Claudio Muñoz C.
	\end{tabular} \\
    Auxiliar de laboratorio:
	& \begin{tabular}[t]{@{}l@{}}
		Nicolás Valenzuela Figueroa  
	\end{tabular} \\
	& \\
	\multicolumn{2}{l}{Fecha de entrega: \today} \\
	\multicolumn{2}{l}{\localizacionuniversidad}
\end{tabular}}{

}

% CONFIGURACIONES
\input{lib/config}
% IMPORTACIÓN DE LIBRERÍAS
\input{lib/env/imports}
% IMPORTACIÓN DE FUNCIONES Y ENTORNOS
\input{lib/cmd/all}
% IMPORTACIÓN DE ESTILOS
\input{lib/style/all}
% CONFIGURACIÓN INICIAL DEL DOCUMENTO
\input{lib/cfg/init}

% INICIO DE LAS PÁGINAS
\begin{document}

% PORTADA
\input{lib/page/portrait}
% CONFIGURACIÓN DE PÁGINA Y ENCABEZADOS
\input{lib/cfg/page}
% RESUMEN O ABSTRACT

% TABLA DE CONTENIDOS - ÍNDICE
\input{lib/page/index} % Índice, se puede borrar
% CONFIGURACIONES FINALES
\input{lib/cfg/final}

% ======================= INICIO DEL DOCUMENTO =======================

% ========== RESUMEN ==========
\begin{resumen}
	% -----  Inicio de resumen -----
Cuerpo 

\end{resumen}

\newpage
% ========== Introducción ========== %
\section{Introducción}
Cuerpo
% ----- Fin de resumen -----



\newpage
% ========== ANÁLISIS DE RESULTADOS ==========
\section{Parte 1. Titulo}

\subsection{Parte 1}





\clearpage 
\subsection{Parte 2}




\clearpage 



\newpage
% ========== DISCUSIÓN ==========
% Discusión (valor 1,5 punto)


\section{Parte 2. Título}
\subsection{Parte a) Titulo}
Cuerpo


\definecolor{codeblue}{HTML}{1F1F2E} 
\definecolor{codegray}{rgb}{0.5,0.5,0.5}
\definecolor{codepurple}{rgb}{0.58,0,0.82}
\definecolor{backcolour}{rgb}{0.95,0.95,0.92}

\lstdefinestyle{mystyle}{
    backgroundcolor=\color{codeblue},
    commentstyle=\color{codegray},
    keywordstyle=\color{magenta},
    numberstyle=\tiny\color{codegray},
    stringstyle=\color{codepurple},
    basicstyle=\ttfamily\color{white},
    breakatwhitespace=false,
    breaklines=true,
    captionpos=b,
    keepspaces=true,
    numbers=left,
    numbersep=5pt,
    showspaces=false,
    showstringspaces=false,
    showtabs=false,
    tabsize=2
}

\clearpage 
\subsection{Parte b) Título }


\clearpage
%%%%%%%%%%%%%%%%%%%%%%%%%%%%%%%%%%%%%%%

\newpage
\input{Conclusion.tex}

\end{document}
