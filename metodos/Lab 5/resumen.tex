\begin{resumen}
    La experiencia realizada en el laboratorio, tuvo como principal tematica la corriente alterna y su comportamiento
    en diferentes contextos como lo son, la emision de este y su proceder en un circuito electrico. \\

    La corriente alterna ("CA" o "AC"), que fue el objeto de analisis en esta ocacion, es aquella corriente cuyos
    parametros varian ciclicamente en terminos de magnitud y direccion. Este fenomeno, la cual comunmente suele ser
    o tener un comportamiento "ciclico" implica la aparicion de nuevas variables de analisis de medicion, en donde se
    agregan nuevos parametros dada a las caracteristicas del fenomeno, entre esas se destaca la utilizacion de nuevas
    medidas como el uso de frecuencias y periodos, caracteristico de fenomenos ciclicos, ademas de ser necesario definir
    nuevos parametros e incluso unidades de medida al incorporarse parametros de estudio que antes eran desconocidos
    como ocurre en los circuitos electricos agregando unidades de medida como el "\it Henry\rm" $H$. \\

    La realizacion del laboratorio (es decir, la experiencia), con el fin de poder realizarse se vio en la necesidad
    plantear objetivos respectivos. Estos objetivos serian en base a aquellos propuestos inicialmente por la
    experiencia, es decir, lograr la familiarizacion con generadores y medidores de corriente alterna (en este caso
    frecuencias) como el osiloscopio, adaptarse a los circuitos electricos estudiado en ocasiones pasadas, pero para
    el caso especifico de corriente alterna, destacando una forma especifica como lo seria aquel compuesto por una
    resistencia y una nueva criterio de observacion e indagacion como lo es la "\it inductancia\rm" ($L$). \\

    De manera sucinta, en base a la utilizacion del equipamiento especifico y necesario en el cual primordiaba el uso
    de un osiloscopio (medidor de oscilaciones) y, un generador de frecuencias (o voltajes alternantes), se realizo
    un estudio del comportamiento en las generacion de frecuencias en casos externos a los circuitos electricos, esto
    con el fin de familiarizarse con el equipamiento. Despues, se conectarian a un circuito con el fin de realizar el
    respectivo analisis en este ultimo ambito, realizando mediciones a modo de examinar el proceder de la corriente en
    contextos estudiados en previas experiencias, empleando ademas, nuevas componentes (caso de la inductancia)
    en este ultimo area o ambito de investigacion. \\

    Dada a las condiciones de la investigacion empleando un nuevo tipo de corriente, pueden generarse conclusiones
    importantes como lo es el caso de el entendimiento en el funcionamiento de la nueva instrumentacion y la
    familiarizacion con el comportamiento de una nueva clase de circuitos y componentes para esta ultima especie
    estudiada.
\end{resumen}


